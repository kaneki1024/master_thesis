\documentclass[dvipdfmx,12pt,a4paper]{jreport}
	\usepackage{geometry}
    \usepackage{pdfpages}
    \usepackage{graphicx}
    \usepackage{newtxmath}
    \usepackage{array}
    \usepackage{siunitx}
    \usepackage{url}
    \usepackage{amsmath}
    \usepackage{comment}
    \usepackage{amsfonts}
    \usepackage{here}
    \geometry{left=20mm,right=20mm,top=25mm,bottom=25mm} %geometry:余白設定
    \renewcommand{\bibname}{参考文献}
    %\renewcommand{\figurename}{Fig.}
    %\renewcommand{\tablename}{Tab.}
    \makeatletter
    \DeclareRobustCommand\cite{\unskip
    	\@ifnextchar[{\@tempswatrue\@citex}{\@tempswafalse\@citex[]}}
    \def\@cite#1#2{$^{\hbox{\scriptsize{[#1\if@tempswa , #2\fi}]}}$}
    \def\@biblabel#1{[#1]}
    \makeatother
    
    \graphicspath{{./graphics/}} %図のディレクトリを設定.フォルダの場所によって先頭' . 'の数を変える.
\begin{document}
	\begin{titlepage}
		
		\begin{center}
			
			\vspace{20truept}
			{\LARGE 2022年度}\\
			\vspace{15truept}
			{\LARGE 修士論文}
			
			\vspace{50truept}
			
			{\Huge Silk Fibroin Filmの圧電性向上の研究}\\
			\vspace{10truept}
			
			\vspace*{280truept}
			
			{\LARGE 1521516}\\
			\vspace{5truept}
			
			{\LARGE 金木 進}\\
			\vspace{60truept}
			
			{\LARGE 2023年3月}
			\vspace{30truept}
			
			{\LARGE 東京理科大学}\\
			\vspace{15truept}
			
			{\LARGE 理学研究科 応用物理学専攻}\\
			\vspace{15truept}
			
			{\LARGE 中嶋研究室}\\
			
		\end{center}
		
		
	\end{titlepage}
  \thispagestyle{empty}
	\clearpage
\addtocounter{page}{0}
\tableofcontents
  \chapter{序論}
		\section{研究背景}
		\section{研究の目的}
	\chapter{原理}
		\section{圧電基本式}
		\section{電気機械結合係数}
	\chapter{実験手法}
		\section{試料作製方法}
		\section{評価方法}
			\subsection{XRD}
			\subsection{PFM}
			\subsection{誘電率測定}
			\subsection{DSC}
	\chapter{結果と考察}
	\chapter{総括}
	\chapter{付録}
\end{document}