\documentclass[dvipdfmx,12pt,a4paper]{jreport}
	\usepackage{geometry}
    \usepackage{pdfpages}
    \usepackage{graphicx}
    \usepackage{newtxmath}
    \usepackage{array}
    \usepackage{siunitx}
    \usepackage{url}
    \usepackage{amsmath}
    \usepackage{comment}
    \usepackage{amsfonts}
    \usepackage{here}
    \geometry{left=20mm,right=20mm,top=25mm,bottom=25mm} %geometry:余白設定
    \renewcommand{\bibname}{参考文献}
    %\renewcommand{\figurename}{Fig.}
    %\renewcommand{\tablename}{Tab.}
    \makeatletter
    \DeclareRobustCommand\cite{\unskip
    	\@ifnextchar[{\@tempswatrue\@citex}{\@tempswafalse\@citex[]}}
    \def\@cite#1#2{$^{\hbox{\scriptsize{[#1\if@tempswa , #2\fi}]}}$}
    \def\@biblabel#1{[#1]}
    \makeatother
    
    \graphicspath{{./graphics/}} %図のディレクトリを設定.フォルダの場所によって先頭' . 'の数を変える.
\begin{document}
	\begin{titlepage}
		
		\begin{center}
			
			\vspace{20truept}
			{\LARGE 2022年度}\\
			\vspace{15truept}
			{\LARGE 修士論文}
			
			\vspace{50truept}
			
			{\Huge Silk Fibroin Filmの圧電性向上の研究}\\
			\vspace{10truept}
			
			\vspace*{280truept}
			
			{\LARGE 1521516}\\
			\vspace{5truept}
			
			{\LARGE 金木 進}\\
			\vspace{60truept}
			
			{\LARGE 2023年3月}
			\vspace{30truept}
			
			{\LARGE 東京理科大学}\\
			\vspace{15truept}
			
			{\LARGE 理学研究科 応用物理学専攻}\\
			\vspace{15truept}
			
			{\LARGE 中嶋研究室}\\
			
		\end{center}
		
		
	\end{titlepage}
  \thispagestyle{empty}
	\clearpage
\addtocounter{page}{0}
\tableofcontents
  \chapter{序論}
		\section{研究背景}
		\section{研究の目的}
	\chapter{原理}
		\section{圧電基本式と電気機械結合係数}
		圧電体には正圧電効果と逆圧電効果という性質が存在する。
		生じた歪みに対して、応力と電場の寄与がある。
		さらに生じた電束密度に対しても電場と歪みの二つの寄与がある。
		これらを式にまとめると
				\begin{eqnarray}
					\begin{cases}
						\delta S=\frac{\partial S}{\partial T}\delta T + \frac{\partial S}{\partial E} \delta E 
						= s^{E}\delta T+d \delta E & \\
						\delta D=\frac{\partial D}{\partial T}\delta T + \frac{\partial D}{\partial E}\delta E 
						= d \delta T+\varepsilon^T \delta E  &
					\end{cases}
					\label{圧電d形式1}
				\end{eqnarray}
			となる。実際の試料は1軸方向、2軸方向、3軸方向のみだけでなく、
			せん断歪みを考慮する必要があり、テンソル形式で記述される。
			ここで、$\delta S\rightarrow S, \delta T\rightarrow T,
			\delta E\rightarrow E, \delta D\rightarrow D$とし、テンソル行列を$\left[ \right]$で表すと
			式\ref{圧電d形式1}は
			\begin{equation}
				\begin{cases}
					\left[S\right]=\left[s^E\right]\left[T\right]+\left[d_t\right]\left[E\right]& \\
					\left[D\right]=\left[d\right]\left[T\right]+\left[\varepsilon^T\right]\left[E\right]
				\end{cases}
				\label{圧電d形式2}
			\end{equation}
			となり、これを圧電d形式と呼ぶ。
			式\ref{圧電d形式2}をテンソルの要素も含めて記述すると
			\begin{equation}
				\begin{cases}
				\left(
				\begin{array}{c}
					S_1 \\
					S_2 \\
					S_3 \\
					S_4 \\
					S_5 \\
					S_6 
				\end{array}
				\right)=\left(
				\begin{array}{cccccc}
				s^E_{11} & s^E_{12} & s^E_{13} & 0 & 0 & 0 \\
				s^E_{21} & s^E_{22} & s^E_{23} & 0 & 0 & 0 \\
				s^E_{13} & s^E_{32} & s^E_{33} & 0 & 0 & 0 \\
				0 & 0 & 0 & s^E_{44} & 0 & 0 \\
				0 & 0 & 0 & 0 & s^E_{44} & 0 \\
				0 & 0 & 0 & 0 & 0 & s^E_{66} 
				\end{array}
				\right)
				\left(
				\begin{array}{c}
					T_1 \\
					T_2 \\
					T_3 \\
					T_4 \\
					T_5 \\
					T_6 
				\end{array}
				\right)+
				\left(
				\begin{array}{ccc}
					0&0	&d_{31} \\
					0&0	&d_{31} \\
					0&0	&d_{33} \\
					0&d_{15}&0 \\
					d_{15}&0&0 \\
					0&0	&0 
				\end{array}
				\right)
				\left(
				\begin{array}{c}
					E_1 \\
					E_2 \\
					E_3 \\
				\end{array}
				\right) & \\
				\left(
				\begin{array}{c}
					D_1 \\
					D_2 \\
					D_3
				\end{array}\right)=\left(
				\begin{array}{cccccc}
					0&0&0&0&d_{15}&0 \\
					0&0&0&d_{15}&0&0 \\
					d_{31}&d_{31}&d_{33}&0&0&0
				\end{array}\right)
				\left(
				\begin{array}{c}
					T_1 \\
					T_2 \\
					T_3 \\
					T_4 \\
					T_5 \\
					T_6 
				\end{array}
				\right)+\left(
				\begin{array}{ccc}
					\varepsilon^T_{11}&0&0 \\
					0&\varepsilon^T_{11}&0 \\
					0&0&\varepsilon^T_{33}
				\end{array}\right)
				\left(
				\begin{array}{c}
					E_1 \\
					E_2 \\
					E_3 \\
				\end{array}\right)
				\end{cases}
			\end{equation}
			となる。また、$s^E_{66}=2\left(s^E_{11}-s^E_{12}\right)$である。
			式\ref{圧電d形式2}を式変形すると
			\begin{equation}
				\begin{cases}
				\left[T\right]=\left[c^E\right]\left[S\right]-\left[e_t\right]\left[E\right] & \\
				\left[D\right]=\left[e\right]\left[S\right]+\left[\varepsilon^S\right]\left[E\right]
				\end{cases}
				\label{圧電e形式}
			\end{equation}
			\begin{equation}
				\begin{cases}
					\left[S\right]=\left[s^D\right]\left[T\right]-\left[g_t\right]\left[D\right] & \\
					\left[E\right]=-\left[g\right]\left[T\right]+\left[\beta^T\right]\left[D\right]
				\end{cases}
				\label{圧電g形式}
			\end{equation}
			\begin{equation}
				\begin{cases}
					\left[T\right]=\left[c^D\right]\left[S\right]-\left[h_t\right]\left[D\right] & \\
					\left[E\right]=-\left[h\right]\left[S\right]+\left[\beta^s\right]\left[D\right]
				\end{cases}
				\label{圧電h形式}
			\end{equation}
			の三式を導くことができ、式\ref{圧電e形式}を圧電e形式, 
			式\ref{圧電g形式}を圧電g形式, 式\ref{圧電h形式}を圧電h形式と呼ぶ。応力$T$, 電場$E$, 歪み$S$, 電束密度$D$の係数である$[d],[e],[g],[h]$ではそれぞれ、
			\begin{equation}
				d_{ij}=\left(\frac{\partial D_i}{\partial T_j}\right)_E=\left(\frac{\partial S_j}{\partial E_i}\right)_T
			\end{equation}
			\begin{equation}
				e_{ij}=\left(\frac{\partial D_i}{\partial S_j}\right)_E=-\left(\frac{\partial T_j}{\partial E_i}\right)_S
			\end{equation} 
			\begin{equation}
				g_{ij}=-\left(\frac{\partial E_i}{\partial T_j}\right)_D=\left(\frac{\partial S_j}{\partial D_i}\right)_T
			\end{equation}
			\begin{equation}
				h_{ij}=-\left(\frac{\partial E_i}{\partial S_j}\right)_D=-\left(\frac{\partial T_j}{\partial D_i}\right)_S
			\end{equation}
			で定義される。また、それぞれの圧電定数間には弾性コンプライアンス$s$、
			誘電率$\varepsilon$を介して以下の関係がある。
			\begin{equation}
				d = e s
			\end{equation}
			\begin{equation}
				g = h s
			\end{equation}
			\begin{equation}
				d = \varepsilon g
			\end{equation}
			\begin{equation}
				e = \varepsilon h
			\end{equation}

			圧電定数と同様に、圧電効果を示す定数として電気機械結合係数$k$がある。
			電気機械結合係数$k$は電気的エネルギーと機械的エネルギーの変換を表す係数であり、
			式\ref{電気機械結合係数の定義}のように与えた電気エネルギーと生じた機械エネルギー、
			あるいは与えた機械的エネルギーと生じた電気的エネルギーの比の平方根で定義される。
			\begin{equation}
			k^2=\frac{\mbox{出力機械的エネルギー}}{\mbox{入力電気的エネルギー}}=
			\frac{\mbox{出力電気的エネルギー}}{\mbox{入力機械的エネルギー}}
			\label{電気機械結合係数の定義}
			\end{equation}
			また、電気機械結合係数は$k$は$k_{31}, k_{33}$の様に、
			電場方向と機械入出力方向を示す二つの下付き文字で表現される。
			表\ref{圧電材料}に代表的な圧電材料であるPZTとPVDFの物性値をまとめた。
			\begin{table}[h]
				\centering
				\caption{PZTとPVDFの物性値\cite{物性値}}
				\label{圧電材料}
				\begin{tabular}{c|ccccc}\hline
					材料&弾性率[N/m$^2$]&比誘電率$\varepsilon/\varepsilon_0$&$d_{31}$[pC/N]&$g_{31}$[Vm/N]&電気機械結合係数$k_{31}$ \\ \hline \hline
					PZT&83.3&1200&110&0.01&0.31 \\
					PVDF&3.0&13&20&0.17&0.10 \\ 
					水晶&77&4.5&2&0.05&0.09 \\
					VDCN/VAC&4.5&5&6&0.13&0.06 \\ 
					VDCN/MMA&4.5&5&0.3&0.007&0.003 \\ \hline
				\end{tabular} 
			\end{table}
	\chapter{実験手法}
		\section{試料作製方法}
		\section{評価方法}
			\subsection{$\theta - 2\theta$測定(XRD)}
			\subsection{Pole Figure測定(XRD)}
			\subsection{PFM}
			\subsection{誘電率測定}
			\subsection{DSC}
	\chapter{結果と考察}
	\chapter{総括}
	\chapter{付録}
\end{document}