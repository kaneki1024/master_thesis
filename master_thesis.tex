\documentclass[dvipdfmx,12pt,a4paper]{jreport}
	\usepackage{geometry}
    \usepackage{pdfpages}
    \usepackage{graphicx}
    \usepackage{newtxmath}
    \usepackage{array}
    \usepackage{siunitx}
    \usepackage{url}
    \usepackage{amsmath}
    \usepackage{comment}
    \usepackage{amsfonts}
		\usepackage{bm}
    \usepackage{here}
    \geometry{left=20mm,right=20mm,top=25mm,bottom=25mm} %geometry:余白設定
    \renewcommand{\bibname}{参考文献}
    %\renewcommand{\figurename}{Fig.}
    %\renewcommand{\tablename}{Tab.}
    \makeatletter
    \DeclareRobustCommand\cite{\unskip
    	\@ifnextchar[{\@tempswatrue\@citex}{\@tempswafalse\@citex[]}}
    \def\@cite#1#2{$^{\hbox{\scriptsize{[#1\if@tempswa , #2\fi}]}}$}
    \def\@biblabel#1{[#1]}
    \makeatother
		\setcounter{secnumdepth}{4}
    
    \graphicspath{{./graphics/}} %図のディレクトリを設定.フォルダの場所によって先頭' . 'の数を変える.
\begin{document}
	\begin{titlepage}
		
		\begin{center}
			
			\vspace{20truept}
			{\LARGE 2022年度}\\
			\vspace{15truept}
			{\LARGE 修士論文}
			
			\vspace{50truept}
			
			{\Huge 複素誘電スペクトロスコピーによる\\絹糸の誘電・圧電・機械的特性の評価}\\
			\vspace{10truept}
			
			\vspace*{280truept}
			
			{\LARGE 1521516}\\
			\vspace{5truept}
			
			{\LARGE 金木 進}\\
			\vspace{60truept}
			
			{\LARGE 2023年3月}
			\vspace{30truept}
			
			{\LARGE 東京理科大学}\\
			\vspace{15truept}
			
			{\LARGE 理学研究科 応用物理学専攻}\\
			\vspace{15truept}
			
			{\LARGE 中嶋研究室}\\
			
		\end{center}
		
		
	\end{titlepage}
  \thispagestyle{empty}
	\clearpage
\addtocounter{page}{0}
\tableofcontents
  \chapter{序論}
		\section{研究背景}
		\subsection{SDGs}
		2015年9月25日-27日、国連総会において「国連持続可能な開発サミット」
		が開催された。そして、2030年までに実現する国際目標として
		 "Transforming Our World: 2030 Agenda for Sustainable Development" が採択された\cite{sdgs_2030}。
		だれも取り残さなさい(leave no one behind)を基本理念とし、
		17の目標と169のターゲットから構成されている。まとめて
		「持続可能な開発目標(SDGs: Sustainable Development Goals)」と呼ばれる。
		図\ref{sdgs_poster}の通り、17個の目標は貧困、紛争、健康などの人間的な活動の問題から
		気候変動、海洋環境など環境問題まで言及している。
		科学技術の向上により人類の生活は向上してきたが、今後は環境に対する負荷、
		発展途上国にも普及するか、なども考慮しなくてはならない。
		\\
		\\

		\begin{figure}[h]
			\centering
			\includegraphics[width=\linewidth]{sdgs_poster.jpg}
			\caption{持続な開発目標(SDGs)における17個の目標が記述されたポスター\cite{sdgs_poster}}
			\label{sdgs_poster}
		\end{figure}
		\newpage
		\subsection{絹糸}
		\subsubsection{絹糸の概論}
		\subsubsection{絹糸の構造}
		図\ref{fibroin}(a)の様に絹糸はフィブロインとセリシンで構成されている。
		絹糸を構成する一つの蛋白質であるフィブロインは3個の分子量の異なる分子により
		構成されている。
		分子量 360 kDaのH鎖と
		27 kDaのL鎖がC末端近傍にてジスルフィド結合したヘテロ2量体が、
		P25の1分子とH鎖の6箇所にて水素結合で繋がっている\cite{fibroin_structure}。
		H鎖は周期構造\cite{H鎖の周期構造}をとる結晶領域とそれを繋ぐ非晶領域が存在する。
		H鎖一本において11個の結晶領域が存在し、
		図\ref{fibroin}(c)のようなアミノ酸配列である。
		結晶領域は最安定状態では逆平行$\beta$シート構造であり、
		結晶構造は単斜晶系をとる\cite{fibroinの結晶構造,marshモデル}。
		これをsilk IIと呼ぶ。
		準安定状態で$\alpha$ helix構造をとり結晶構造は斜方晶系をとる\cite{fibroinの結晶構造,fibroin_準安定状態}。
		これをsilk Iと呼ぶ。加熱や延伸によりsilk Iはsilk IIへ転移する。
		薄膜の気液界面にてthreefoldヘリックスを形成し、silk IIIと呼ぶ\cite{silk_III}。

		\begin{figure}[h]
			\centering
			\includegraphics[width=\linewidth]{fibroin_structure.jpg}
			\caption{フィブロインの構成。(a):絹糸の構造, (b):フィブロインにおけるH鎖とL鎖の関係, 
			(c):H鎖周期構造におけるアミノ酸配列。}
			\label{fibroin}
		\end{figure}

		\newpage
		\section{研究の目的}

	\chapter{原理}
		\section{圧電性}
		\subsection{圧電基本式}
		圧電体には正圧電効果と逆圧電効果という性質が存在する。
		生じた歪みに対して、応力と電場の寄与がある。
		さらに生じた電束密度に対しても応力と電場の二つの寄与がある。
		これらを式にまとめると
				\begin{eqnarray}
					\begin{cases}
						\delta S=\frac{\partial S}{\partial T}\delta T + \frac{\partial S}{\partial E} \delta E 
						= s^{E}\delta T+d \delta E & \\
						\delta D=\frac{\partial D}{\partial T}\delta T + \frac{\partial D}{\partial E}\delta E 
						= d \delta T+\varepsilon^T \delta E  &
					\end{cases}
					\label{圧電d形式1}
				\end{eqnarray}
			となる。実際の試料は1軸方向、2軸方向、3軸方向のみだけでなく、
			せん断歪みを考慮する必要があり、テンソル形式で記述される。
			ここで、$\delta S\rightarrow S, \delta T\rightarrow T,
			\delta E\rightarrow E, \delta D\rightarrow D$とし、テンソル行列を$\left[ \right]$で表すと
			式\ref{圧電d形式1}は
			\begin{equation}
				\begin{cases}
					\left[S\right]=\left[s^E\right]\left[T\right]+\left[d_t\right]\left[E\right]& \\
					\left[D\right]=\left[d\right]\left[T\right]+\left[\varepsilon^T\right]\left[E\right]
				\end{cases}
				\label{圧電d形式2}
			\end{equation}
			となり、これを圧電d形式と呼ぶ。
			式\ref{圧電d形式2}をテンソルの要素も含めて記述すると
			\begin{equation}
				\begin{cases}
				\left(
				\begin{array}{c}
					S_1 \\
					S_2 \\
					S_3 \\
					S_4 \\
					S_5 \\
					S_6 
				\end{array}
				\right)=\left(
				\begin{array}{cccccc}
				s^E_{11} & s^E_{12} & s^E_{13} & s^E_{14} & s^E_{15} & s^E_{16} \\
				s^E_{21} & s^E_{22} & s^E_{23} & s^E_{24} & s^E_{25} & s^E_{26} \\
				s^E_{31} & s^E_{32} & s^E_{33} & s^E_{34} & s^E_{35} & s^E_{36} \\
				s^E_{41} & s^E_{42} & s^E_{43} & s^E_{44} & s^E_{45} & s^E_{46} \\
				s^E_{51} & s^E_{52} & s^E_{53} & s^E_{54} & s^E_{55} & s^E_{56} \\
				s^E_{61} & s^E_{62} & s^E_{63} & s^E_{64} & s^E_{65} & s^E_{66} \\
				\end{array}
				\right)
				\left(
				\begin{array}{c}
					T_1 \\
					T_2 \\
					T_3 \\
					T_4 \\
					T_5 \\
					T_6 
				\end{array}
				\right)+
				\left(
				\begin{array}{ccc}
					d_{11} & d_{21}	& d_{31} \\
					d_{12} & d_{22}	& d_{32} \\
					d_{13} & d_{32}	& d_{33} \\
					d_{14} & d_{42} & d_{34} \\
					d_{15} & d_{52} & d_{35} \\
					d_{16} & d_{62}	& d_{36}
				\end{array}
				\right)
				\left(
				\begin{array}{c}
					E_1 \\
					E_2 \\
					E_3 \\
				\end{array}
				\right) & \\
				\left(
				\begin{array}{c}
					D_1 \\
					D_2 \\
					D_3
				\end{array}\right)=\left(
				\begin{array}{cccccc}
					d_{11} & d_{12} & d_{13} & d_{14} & d_{15} & d_{16} \\
					d_{21} & d_{22} & d_{23} & d_{24} & d_{25} & d_{26} \\
					d_{31} & d_{32} & d_{33} & d_{34} & d_{35} & d_{36}
				\end{array}\right)
				\left(
				\begin{array}{c}
					T_1 \\
					T_2 \\
					T_3 \\
					T_4 \\
					T_5 \\
					T_6 
				\end{array}
				\right)+\left(
				\begin{array}{ccc}
					\varepsilon^T_{11}&0&0 \\
					0&\varepsilon^T_{11}&0 \\
					0&0&\varepsilon^T_{33}
				\end{array}\right)
				\left(
				\begin{array}{c}
					E_1 \\
					E_2 \\
					E_3 \\
				\end{array}\right)
				\end{cases}
				\label{圧電d形式_テンソル}
			\end{equation}
			となる。式\ref{圧電d形式2}を式変形すると
			\begin{equation}
				\begin{cases}
				\left[T\right]=\left[c^E\right]\left[S\right]-\left[e_t\right]\left[E\right] & \\
				\left[D\right]=\left[e\right]\left[S\right]+\left[\varepsilon^S\right]\left[E\right]
				\end{cases}
				\label{圧電e形式}
			\end{equation}
			\begin{equation}
				\begin{cases}
					\left[S\right]=\left[s^D\right]\left[T\right]-\left[g_t\right]\left[D\right] & \\
					\left[E\right]=-\left[g\right]\left[T\right]+\left[\beta^T\right]\left[D\right]
				\end{cases}
				\label{圧電g形式}
			\end{equation}
			\begin{equation}
				\begin{cases}
					\left[T\right]=\left[c^D\right]\left[S\right]-\left[h_t\right]\left[D\right] & \\
					\left[E\right]=-\left[h\right]\left[S\right]+\left[\beta^s\right]\left[D\right]
				\end{cases}
				\label{圧電h形式}
			\end{equation}
			の三式を導け、それぞれ式\ref{圧電e形式}を圧電e形式, 
			式\ref{圧電g形式}を圧電g形式, 式\ref{圧電h形式}を圧電h形式と呼ぶ。応力$T$, 電場$E$, 歪み$S$, 電束密度$D$の係数である$[d],[e],[g],[h]$ではそれぞれ、
			\begin{equation}
				d_{ij}=\left(\frac{\partial D_i}{\partial T_j}\right)_E=\left(\frac{\partial S_j}{\partial E_i}\right)_T
			\end{equation}
			\begin{equation}
				e_{ij}=\left(\frac{\partial D_i}{\partial S_j}\right)_E=-\left(\frac{\partial T_j}{\partial E_i}\right)_S
			\end{equation} 
			\begin{equation}
				g_{ij}=-\left(\frac{\partial E_i}{\partial T_j}\right)_D=\left(\frac{\partial S_j}{\partial D_i}\right)_T
			\end{equation}
			\begin{equation}
				h_{ij}=-\left(\frac{\partial E_i}{\partial S_j}\right)_D=-\left(\frac{\partial T_j}{\partial D_i}\right)_S
			\end{equation}
			で定義される。また、それぞれの圧電定数間には弾性コンプライアンス$s$、
			誘電率$\varepsilon$を介して以下の関係がある。
			\begin{equation}
				d = e s
			\end{equation}
			\begin{equation}
				g = h s
			\end{equation}
			\begin{equation}
				d = \varepsilon g
			\end{equation}
			\begin{equation}
				e = \varepsilon h
			\end{equation}
			$d, e, g, h$はどれも式\ref{圧電d形式_テンソル}の様にテンソルで記述される。
			具体的なテンソルの中身は結晶と試料の全体の対称性から決まる。
			結晶が無秩序に存在する場合、分子や結晶の対称性に関係なく圧電性は現れない。
			結晶が無秩序に存在する状態は、$\infty$個の$\infty$回軸がある。
			延伸やポーリングなどの処理により$\infty$回軸を減らす。
			圧電性や焦電性を示すマクロな対称性においては$\infty (C_\infty), \infty$ mm
			($C_{\infty v}$), $\infty 2(D_\infty)$の3群となる。
			シルクフィブロインは$D_\infty$となり圧電d行列は以下のように決まる。
			\begin{equation}
				\left(
				\begin{array}{cccccc}
					0 & 0 & 0 & d_{14} & 0 & 0 \\
					0 & 0 & 0 & 0 & -d_{14} & 0 \\
					0 & 0 & 0 & 0 & 0 & 0
				\end{array}\right)
			\end{equation}
			\subsection{電気機械結合係数}
			物理変数が異なる系の物理変数を変化させる現象を結合効果という。
			力や温度変化によって分極が変化する現象である圧電効果(piezoelectric effect)
			及び焦電効果(pyroelectric effect)は結合効果の一つである。

			\begin{figure}[h]
				\centering
				\includegraphics{結合効果.jpg}
				\caption{線形な結合効果}
				\label{結合効果}
			\end{figure}
			図\ref{結合効果}のように2種類の共役な物理変数 ($x_1$, $X_1$),($x_2$, $X_2$)
			について、その間に線形な結合があるとする。
			熱力学ポテンシャルは、一般化したひずみ $x$ を独立変数として
			式\ref{線形結合効果の熱力学ポテンシャル}のように記述出来る。
			\begin{equation}
				\Phi=\frac{1}{2}c_{11}x_1^2 + c_{12}x_1x_2+\frac{1}{2}c_{22}x_2^2
				\label{線形結合効果の熱力学ポテンシャル}
			\end{equation}
			$x$に共役な物理変数$X$は
			\begin{equation}
				X_1 = \left( \frac{\partial \Phi}{\partial x_1} \right)_{x_2} = c_{11}x_1+c_{12}x_2
			\end{equation}
			\begin{equation}
				X_2 = \left( \frac{\partial \Phi}{\partial x_2} \right)_{x_1} = c_{12}x_1 + c_{22}x_2
			\end{equation}
			となり、係数 $c_{12}$を介して異なる物理変数への結合が起きることが分かる。
			非線形な結合である場合も線形な場合と同様の上式2つで記述できる。
			両方向の結合の係数が等しく、これを相反定理($c_{12}=c_{21}$)という。
			また、異なる系の間におけるエネルギー変換効率として結合系数$k$を式\ref{結合係数}のように定義する。
			\begin{equation}
				k^2=\frac{c_{12}^2}{c_{11}c_{22}}
				\label{結合係数}
			\end{equation}
			
			例えば共役な変数($x_1$, $X_1$)として応力$T$とひずみ$S$を採用する。
			また、もう一つの共役な物理変数($x_2$, $X_2$)として電場$E$と電束密度$D$を採用する。
			その二つの共役な物理変数を結ぶ結合の係数を$d$とおくとき、熱力学ポテンシャル$\Phi$は以下のように書ける。
			\begin{equation}
				\Phi = \frac{1}{2}s^ET^2 + d T E + \frac{1}{2}\varepsilon^T E^2
			\end{equation}
			一般化した議論と同様に熱力学ポテンシャル$\Phi$から$X_1=S$, $X_2=D$を計算する
			と以下のように圧電d形式を入手できる。
			\begin{equation}
				S = \left(\frac{\partial \Phi}{\partial T} \right)_E = s^E T + d E
			\end{equation}
			\begin{equation}
				D=\left(\frac{\partial \Phi}{\partial E}\right)_T = dT+\varepsilon^T E
			\end{equation}
			結合係数$k$は以下のように記述でき、電気的特性と機械的特性の変換であるため電気機械結合係数と呼ばれる。
			\begin{equation}
				k^2 = \frac{d^2}{\varepsilon^Ts^E}
				\label{電気機械結合係数d}
			\end{equation}

			自由境界条件($T=0$)で逆圧電効果を例に単位体積当たりの入力のエネルギーと出力のエネルギーを計算する。
			圧電d形式、式\ref{圧電d形式1}から自由状態($T=0$)において電源から
			\begin{equation}
				D=\varepsilon^T E
			\end{equation}
			という電束密度が供給される。これより、入力された電気エネルギー$U_{in}$は以下のように計算される。
			\begin{equation}
				U_{in} = \int^{D}_0 E dD = \int^{D}_0 \frac{D}{\varepsilon^T} dD  
				= \frac{1}{2}\frac{D^2}{\varepsilon^T}
				= \frac{1}{2}\varepsilon^T E^2 
			\end{equation}
			圧電体においては、入力された電気エネルギー$U_{in}$の一部が機械エネルギーに変換される。
			電場に対するひずみは圧電d形式、式\ref{圧電d形式1}において$T=0$として以下の通りである。
			\begin{equation}
				S = d E
			\end{equation}
			単位体積あたりの機械エネルギーは
			\begin{equation}
				U_{out} = \int^S_0 \frac{S}{s^E} d S = \frac{1}{2} \frac{1}{s^E}S^2 
				= \frac{1}{2}\frac{d^2}{s^E}E^2
			\end{equation}
			となる。よって入力電気エネルギーと出力機械エネルギーの割合は以下のように、式\ref{電気機械結合係数d}
			と同等の電気機械結合係数を得る。
			\begin{equation}
				k^2 = \frac{\mbox{出力機械的エネルギー}}{\mbox{入力電気的エネルギー}}
				= \frac{U_{out}}{U_{in}} = \frac{\frac{1}{2}\varepsilon^T E^2}{\frac{1}{2}\frac{d^2}{s^E}E^2}
				= \frac{d^2}{\varepsilon^T s^E}
			\end{equation}
			また、正圧電効果も同様に議論でき以下の関係が言える。
			\begin{equation}
				k^2=\frac{\mbox{出力機械的エネルギー}}{\mbox{入力電気的エネルギー}}=
				\frac{\mbox{出力電気的エネルギー}}{\mbox{入力機械的エネルギー}}
				\label{電気機械結合係数の定義}
			\end{equation}
			電気機械結合係数の二乗$k^2$がエネルギーの変換効率となる。
			$1-k^2$は正圧電効果では機械エネルギーが電気エネルギーに
			変換されなかったエネルギーの総エネルギーに対する割合であり、
			逆圧電効果では材料に蓄えられたエネルギーの総エネルギーに対する割合となる。

			また、$1-k^2$は応力$T$を0とした誘電率$\varepsilon^T$と歪み$S$を0としたときの
			誘電率$\varepsilon^S$の比とも解釈される。
			式\ref{圧電d形式1}の圧電d形式の2式目を以下のように応力$T$を求める形に式変形する。			
			\begin{equation}
				T = \frac{D-\varepsilon^T E}{d}
			\end{equation}
			上式を式\ref{圧電d形式1}の1式目に代入し、電束密度$D$を求める形に変形する。
			この際に式\ref{電気機械結合係数d}を使用した。
			また、圧電e形式である式\ref{圧電e形式}の2式目と等価となる。
			\begin{equation}
				D = \frac{d}{s^E}S + \varepsilon^T\left(1-k^2\right)E
				  = e S + \varepsilon^S E
			\end{equation}
			上式において発生する歪み$S$を0としたとき、以下の関係が導ける。
			\begin{equation}
				\frac{\varepsilon^S}{\varepsilon^T}=1-k^2
			\end{equation}
			インピーダンスアナライザなどを用いて容量を計測し、印加電圧の周波数を変動させながら
			圧電共鳴を計測したとする。圧電共鳴が発生する周波数$f_R$より低周波側は
			自由端条件であり、高周波側は固定端条件となる。
			自由端条件は応力$T$を0とした誘電率$\varepsilon^T$、
			固定端条件では歪み$S$を0とした誘電率$\varepsilon^S$が対応する。
			容量から誘電率を計算するが、以下の式では誘電率の比率であるため
			試料寸法の計測エラーが消える。
			\begin{comment}
			表\ref{圧電材料}に代表的な圧電材料であるPZTとPVDFの物性値をまとめた。
			\begin{table}[h]
				\centering
				\caption{PZTとPVDFの物性値}
				\label{圧電材料}
				\begin{tabular}{c|ccccc}\hline
					材料&弾性率[N/m$^2$]&比誘電率$\varepsilon/\varepsilon_0$&$d_{31}$[pC/N]&$g_{31}$[Vm/N]&電気機械結合係数$k_{31}$ \\ \hline \hline
					PZT&83.3&1200&110&0.01&0.31 \\
					PVDF&3.0&13&20&0.17&0.10 \\ 
					水晶&77&4.5&2&0.05&0.09 \\
					VDCN/VAc&4.5&5&6&0.13&0.06 \\ 
					VDCN/MMA&4.5&5&0.3&0.007&0.003 \\ \hline
				\end{tabular} 
			\end{table}
			\end{comment}
			\newpage
			\subsection{生体材料の圧電性}
			生体材料における最初の圧電性の報告は、1941年に報告されたMartinによる羊毛を用いた検証とされる。
			羊毛を並べて糊を用いて束にする。束を輪切りにして、断面に力を加えると1 V発生したと報告している。
			同時に焦電性の検証もしており、液体窒素に漬けて温度を下げると約10 Vの
			電圧が発生したと報告している\cite{martin1941}。

			最初に研究が活発になった自然由来の材料として木材が挙げられる。Shubnikovによってその圧電性の予測がされた。
			また、非晶質内で結晶粒子が配向している系をpiezoelectric textureと呼んだ\cite{shubnikov1946}。
			Bazhenov と Konstantinovaによって木材の圧電性の実験が行われた\cite{bazhenov1950}。
			その後、Fukadaによって木材のずり圧電効果の正圧電性と逆圧電性の実験が成功し、
			同時に熱力学的証明を行った\cite{fukada_wood}。
			その後、様々な種類の木材における圧電性など詳細な研究が行われた。その結果、
			木材の対称性は斜方晶系$D_2(222)$に属し、$d_{36}$が有限な値を持ち、$|d_{14}|>|d_{36}|$
			であると示された\cite{平井_木材}。

			蛋白質はアミノ酸が図\ref{アルファヘリックス}(A)の様なペプチド結合(CONH)にて結合した構造を持ち、
			ポリペプチドとも呼ばれる。自然由来の材料、合成材料のどちらにおいても
			ポリペプチド化合物における圧電性の研究が行われてきた。
			自然由来の材料ではコラーゲン\cite{fukada_コラーゲン}、絹\cite{fukada_silk_fiber}、DNA\cite{DNA_圧電性}、皮膚\cite{皮膚_圧電性}、
			骨\cite{hoda_bone_1953,fukada_bone}、腱\cite{fukada_コラーゲン}、
			血管壁\cite{動脈_圧電性}、筋肉\cite{筋肉_圧電性}、貝殻\cite{貝_強誘電性_圧電性}
			においてずりの圧電性があると証明された。
			また、動脈\cite{動脈_強誘電性}や貝\cite{貝_強誘電性_圧電性}には圧電性だけでなく、強誘電性も確認されている。
			合成材料においてはpoly-L-alanine, poly-$\gamma$-methyl-L-glutamate, poly-$\gamma$-benzyl-L-glutamate
			などにおいてもずり圧電が確認されている\cite{合成キラル高分子の圧電性}。

			ずり圧電は高分子における不斉炭素が試料の非対称性に寄与している。
			不斉炭素原子にはL体とD体が存在し、その両方が存在するラセミ体などの材料は統計的鏡映対称性が存在するため、
			ずりの圧電性は生まれない。
			生物を構成するアミノ酸はL体しか存在せず、生物由来の糖はD体のみしか存在しない。
			これを生命のホモキラリティと呼ぶ。
			このようにL体もしくはD体のみ存在する場合、ずりの圧電性が生まれる。
			ずりの圧電性である$d_{14}$の符号はL体で構成されている材料においては負であり、
			D体で構成されている材料は正となる。分子の構造が$\alpha$ヘリックス、
			もしくは引き延ばされた$\beta$型分子のどちらにおいても成立する。
			ただし、$\alpha$ヘリックスは双極子の向きが揃っているため、
			分子の歪みにより大きな分極が発生する可能性があり、圧電性においては
			実験及び理論においても研究が進んでいる。
			\begin{figure}[h]
				\centering
				\includegraphics[scale=0.8]{アルファヘリックス_CONH.jpg}
				\caption{ペプチド結合と$\alpha$ヘリックス\cite{CONH_theory}。
				(A):ペプチド結合、(B):$\alpha$ヘリックス内のCONH双極子の内部回転}
				\label{アルファヘリックス}
			\end{figure}
			ポリペプチドの$\alpha$ヘリックスの圧電性は分子格子力学を用いて計算されている\cite{CONH_theory}。
			ヘリックス骨格の圧電率は$6.3$ pC/Nと報告されており、結晶化度を考慮した
			PBLGの実験値と一致している。
			
			$\alpha$ヘリックスの圧電性はCONH内部の双極子回転で生じるが、
			最適化された構造の材料としてL体のポリ乳酸がPLLAが挙げられる。
			PLLAは植物由来の合成材料であり、生分解性が高い。
			また、ポリフッ化ビニリデン(PVDF)に匹敵する圧電性を有し、
			加工が容易であり、焦電性が存在せず圧力以外のセンシングを排除できるため
			工業的に注目されている。特に圧電体でありながら焦電性を有さないという点においては
			キネティック系のHMIへの応用として非常に有利である。
			フィルム状のPLLAを利用したデバイスとして図\ref{リーフグリップリモコン}
			の様なリーフグリップリモコンがある\cite{赤本,村田製作所_PLLA}。PLLAの変形から
			デバイスの曲げとねじれの検知が行う。
			\begin{figure}[h]
				\centering
				\includegraphics[scale=0.9]{リーフグリップリモコン.jpg}
				\caption{PLLAのずり圧電を利用したリーフグリップリモコン\cite{村田製作所_PLLA}}
				\label{リーフグリップリモコン}
			\end{figure}
			\\
			また、PLLAをfiber状に加工してデバイスに組み込んだ例も報告されている。
			図\ref{plla_fiber_device}(A)はPLLAをfiber状からバネ型に加工し、電極を載せたデバイスである。
			バネの伸縮に応じた電圧が計測されている\cite{plla_バネ}。また、図\ref{PLLA_fiber_device}(B)はfiber状のPLLAを
			組み紐に組み込むことでファッショナブルな身体計測デバイスを作成し、脈波などの身体情報を入手した
			と報告している\cite{組み紐_plla}。
			\begin{figure}[h]
				\centering
				\includegraphics{plla_fiber_device.jpg}
				\caption{PLLAをファイバー状に加工してデバイスに組み込んだ事例。
				(A):PLLAをバネ上に加工し振動デバイスを作成\cite{plla_バネ}、(B)PLLAを利用した組み紐型、身体情報計測デバイス\cite{組み紐_plla}}
				\label{PLLA_fiber_device}
			\end{figure}
			\newpage
			$\alpha$ヘリックス構造の分子を持ち、大きな圧電性が期待できるPLLAは理論、実験、
			デバイス応用まで進んでいる。しかし、蛋白質の二次構造として$\alpha$ヘリックス以外にも
			$\beta$ シート構造も取りうる。デバイス応用まで社会実装は進んでいないが$\beta$シート構造も
			圧電性が生じると報告されている\cite{beta_sheet_piezo_theory}。


			\newpage
			\subsection{絹糸の圧電性}
			絹糸の圧電性に関して初めて言及されたのはHarveyとされている\cite{Harvey}。
			ただし、定量的な測定などが行われたのではなく、その可能性に言及されたのみに留まる。
			初めて定量的に評価を行ったのはFukadaである\cite{fukada_silk_fiber}。
			乾燥させた絹糸を圧縮させて45°カットで切り出し、圧電性を評価した。
			$d$定数がおおよそ1 pC/N程度と報告した。
			また、絹糸に含まれるセリシンの除去を行うと圧電性の向上が期待できるとも述べられた。
			その後、Yucelらによって絹糸のセリシンを除去しフィブロインのキャストフィルムを作製して
			圧電性の報告がされた\cite{fibroin_cast_film_piezo}。
			図\ref{silk_fibroin_cast_装置}の(A)の様に190℃のヒートブロックを
			局所的に試料を温めながら、試料をなぞる様に移動させる。同時に力を加えて延伸する。
			硬くて脆いという特徴があるfibroinのキャストフィルムだが2.7倍まで延伸したと報告されている。
			図\ref{silk_fibroin_cast_装置}(B)の動的粘弾性装置(Dynamic Mechanical Analysis: DMA)
			を加工してずり圧電の圧電性を評価した。延伸倍率と圧電$d_{14}$の関係は図\ref{cast_film_data}(A)
			の通りである。試料の切り出しの角度と$d_{14}$の関係は図\ref{cast_film_data}(B)の通りである。
			ずり圧電の特徴である角度依存性が確認出来る。
			\begin{figure}[h]
				\centering
				\includegraphics[scale=0.6]{fibroin_piezo_装置.jpg}
				\caption{Fibroin cast filmの延伸と測定装置\cite{fibroin_cast_film_piezo}。
				(A): 延伸の様子、(B):DMA装置を加工したずり圧電の測定装置。}
				\label{silk_fibroin_cast_装置}
			\end{figure}
			\begin{figure}[h]
				\centering
				\includegraphics[scale=0.6]{cast_film_piezo_data.jpg}
				\caption{Fibroin cast film における圧電性\cite{fibroin_cast_film_piezo}。
				(A):延伸倍率と$d_{14}$定数の関係、(B):$d_{14}$定数の角度依存性。}
				\label{cast_film_data}
			\end{figure}
			\\
			また、FTIRやXRDのデータを合わせて検証した結果、
			図\ref{orientation_beta_sheet_piezo}のように
			配向性と逆平行$\beta$ sheet の含有量の両方を高める
			と圧電性が向上すると述べている。
			\begin{figure}[h]
				\centering
				\includegraphics[scale=0.6]{orientation_beta_sheet_piezo.jpg}
				\caption{配向性、逆平行$\beta$ sheetの含有量と圧電性の関係\cite{fibroin_cast_film_piezo}}
				\label{orientation_beta_sheet_piezo}
			\end{figure}
			\\
			
			Fiber状で作製し、圧電性を検証した事例としてエレクトロスピニング法(電界紡糸法)による報告がある\cite{エレクトロスパン}。
			エレクトロスピニング法(電界紡糸法)とは、
			糸ノズル内のポリマー溶液に高電圧を加えて、ナノファイバーを生成する製法である。
			エレクトロスピニング法を活用してFibroin溶液からFibroin nano fiberを作製し、その圧電性を
			DMA装置やPFMを用いて報告している。
			\begin{figure}[h]
				\centering
				\includegraphics[scale=1]{electrospun_data.jpg}
				\caption{エレクトロスピニング法によるFibroin nano fiber\cite{エレクトロスパン}。(A):SEM像、(B):PFMを用いたバタフライ曲線}
				\label{electrospun}
			\end{figure}

			\newpage
		\section{誘電性}
		電場が印加されると分極を生じる性質を誘電性といい、誘電性を有する物質を誘電体と呼ぶ。
		また、このとき生じる分極を誘電分極と呼ぶ。
		圧電体は誘電体に内包される関係である。
		中心対称性がない誘電体は圧電性を示す誘電体であり、
		中心対称性が存在する誘電体は圧電性を示さない。
		よって圧電体は誘電体としての特性も有する。
		時間変動する電場に対する誘電体の特性を紹介する。
			\subsection{誘電応答における時間的な遅れ}
			時間とともに変動する電場に対して、誘電応答は時間的に遅れて生じる。
			これを誘電緩和現象という。
			電場が角周波数$\omega$で正弦波的に変動すると式\ref{E(t)}の様に表現される。
			\begin{equation}
				E(t)=E_0 e^{i\omega t}
				\label{E(t)}
			\end{equation}
			電束密度は振幅$D_0$、位相が$\delta$遅れた信号として観測されると式\ref{D(t)}の様に表現される。
			\begin{equation}
				D(t)=D_0 e^{i(\omega t-\delta)}
				\label{D(t)}
			\end{equation}
			誘電率$\varepsilon$は電束密度$D$と電場$E$の比例係数として定義されるため式\ref{E(t)}と式\ref{D(t)}
			の比は以下のように計算される。
			\begin{equation}
				\frac{D(t)}{E(t)}=\frac{D_0e^{i(\omega t-\delta)}}{E_0 e^{i\omega t}}
				=\frac{D_0}{E_0}\left(\cos\delta-i\sin\delta\right)=\varepsilon'(\omega)-i\varepsilon''(\omega)
			\end{equation}
			誘電率実部$\varepsilon'$は加えた電場が電気エネルギーとして蓄えられる成分に対応し、貯蔵誘電率とも呼ばれる。
			誘電率虚部$\varepsilon''$は熱エネルギーとなって散逸する成分に対応し、損失誘電率と呼ばれる。
			誘電率実部$\varepsilon'$と誘電率虚部$\varepsilon''$の比は以下の通りであり、損失正接(loss tanget)、
			$\delta$を損失角と呼ぶ。
			\begin{equation}
				\frac{\varepsilon''}{\varepsilon'}=\tan\delta
				\label{tandel}
			\end{equation}
			時間的なずれ($\delta$)の物理的な要因は分極の原因となる電荷の変位や双極子の回転が起こるときに、
			粘性抵抗や質量の効果が働くといった点が挙げられる。その結果、誘電率は測定する周波数に依存して分散現象を示し、
			分極の機構によってさまざまなスペクトルを描く。
			\subsection{誘電性と導電性の関係}
			誘電体は直流におけるインピーダンスの高さから絶縁体の一種とされている。
			しかし交流波においては、周波数が高くなるほどインピーダンスが小さくなり導電性が高くなる。
			電束密度$D$の時間微分は電流密度$J$であり、式\ref{D(t)}を用いて計算すると以下のようになる。
			\begin{equation}
				J=\frac{\partial D}{\partial t}=i\omega D
				\label{電流密度}
			\end{equation}
			オームの法則は以下の通りである。
			\begin{equation}
				J=\sigma E
				\label{オームの法則}
			\end{equation}
			式\ref{電流密度}と式\ref{オームの法則}は等価であるため、
			誘電率の実部$\varepsilon'$虚部$\varepsilon''$と
			導電率の実部$\sigma'$虚部$\sigma''$において
			以下の関係が得られる。
			\begin{equation}
				\sigma'=\omega\varepsilon''
			\end{equation}
			\begin{equation}
				\sigma''=\omega\varepsilon'
			\end{equation}
			上二式から式\ref{tandel}の損失正接の大小が導電性の大小に対応すると理解できる。
			\subsection{周波数応答関数とパルス応答関数}
			ステップ状の電場$E_s$を時間$t=0$において印加したときの電束密度の時間変化を$D_s(t)$とする。
			誘電率は以下のように表現され、誘電余効関数と呼ぶ。
			\begin{equation}
				\varepsilon(t)=\frac{D_s(t)}{E_s}
			\end{equation}
			パルス状の$\delta$関数がステップ関数の一回微分であるため、パルス状の電場を印加したときにおける
			誘電率の時間応答関数は以下のように、誘電余効関数の一回微分となる。また、$\varepsilon_p(t)$をパルス応答関数と呼ぶ。
			\begin{equation}
				\varepsilon_p(t)=\frac{d\varepsilon(t)}{dt}
				\label{パルス応答関数}
			\end{equation}
			応答が刺激により生じるという因果性、一定の刺激に対して一定の応答が生じるという定常性、
			複数の刺激に対する応答は、それぞれの刺激に対する応答の和で与えられるという線形生を仮定する。
			任意の電場$E(t)$に対して生じる電束密度$D(t)$はパルス応答関数を用いて以下のように表現される。
			\begin{equation}
				D(t)=\int^{\infty}_0 \varepsilon_p(t_1)E(t-t_1) dt_1
				\label{ボルツマン重ね合わせの原理}
			\end{equation}
			時間$t$に表れる電束密度という応答が時間$t_1$に与えたパルス刺激の応答の重ね合わせになっている。
			これをボルツマン重ね合わせの原理という。正弦波の電場を印加したとして式\ref{E(t)}を式\ref{ボルツマン重ね合わせの原理}
			に代入する。
			\begin{equation}
				D(t)=\int^{\infty}_0 \varepsilon_p(t_1)E_0e^{i\omega(t-t_1)}dt_1=E(t)\int^{\infty}_0 \varepsilon_p(t_1)e^{-i\omega t_1}dt_1
			\end{equation}
			誘電率は以下のように計算される。
			\begin{equation}
				\varepsilon(t)=\frac{D(t)}{E(t)}=\int^{\infty}_0 \varepsilon_p(t_1)e^{-i\omega t_1} dt_1
			\end{equation}
			$t_1\rightarrow t$にすると上式の最右辺はフーリエ変換の式となる。
			周波数応答関数とパルス応答関数が互いにフーリエ変換の関係であると分かる。
			\begin{equation}
				\varepsilon(\omega)=\int^{\infty}_0 \varepsilon_p(t)e^{i\omega t}dt
				\label{周波数応答関数とパルス応答関数のフーリエ変換}
			\end{equation}
			\subsection{Havriliak-Negamiの式}
			高分子内の双極子の回転緩和を簡単のためモーメント$\mu$の$N$個の双極子が二つの互いに逆向きの
			状態(1,2)のみをとり得るとする。
			それぞれの占有数を$N_1, N_2$とし、状態1から2への遷移確率$\omega_{12}$、
			状態2から状態1への遷移確率$\omega_{21}$としその時間変化は以下のように記述される。
			\begin{equation}
				\frac{dN_1}{dt}=-N_1\omega_{12}+N_2\omega_{21}
			\end{equation}
			\begin{equation}
				\frac{dN_2}{dt}=N_1\omega_{12}-N_2\omega_{21}
			\end{equation}
			遷移確率$\omega_{12}, \omega_{21}$は遷移する際の障壁の高さを$\Delta U$、
			高温極限での遷移確率を$\Gamma$とすると以下のように表される。
			\begin{equation}
				\omega_{12}=\Gamma \exp\left(-\frac{\Delta U+\mu E}{kT} \right)
			\end{equation}
			\begin{equation}
				\omega_{21}=\Gamma \exp\left(- \frac{\Delta U-\mu E}{kT} \right)
			\end{equation}
			双極子の数が$N=N_1+N_2$、分極$P=(N_1-N_2)\mu$であるため分極$P$の時間変動は以下のように計算される。
			\begin{equation}
				\frac{dP}{dt} = \left( \frac{dN_1}{dt}-\frac{dN_2}{dt}\right)\mu
				=2\Gamma e^{-\frac{\Delta U}{kT}}\left(-N_1 e^{-\frac{\mu E}{kT}}+N_2e^{\frac{\mu E}{kT}} \right)\mu
			\end{equation}
			$e^{\frac{\Delta U}{kT}}/2\Gamma=\tau$と置くと以下のようにまとめられる。
			\begin{equation}
				\tau \frac{dP}{dt}=N\mu \sinh\left( \frac{\mu E}{kT}\right) - P \cosh\left( \frac{\mu E}{kT}\right)
			\end{equation}
			電場中の双極子がもつポテンシャルエネルギーが熱エネルギーより低い($\mu E/kT \ll 1$)ため以下の近似が成立する。
			\begin{equation}
				\sinh \left(\frac{\mu E}{kT} \right)\rightarrow \frac{\mu E}{kT}, \ \ \ \ \cosh\left(\frac{\mu E}{kT}\right)\rightarrow 1
			\end{equation}
			近似を用いると分極$P$の時間変化は次のように線形の微分方程式に従う。
			\begin{equation}
				\tau \frac{dP}{dt}+P=\frac{N\mu^2}{kT}E
			\end{equation}
			階段波電場が印加されたとして、上の微分方程式を解くと以下のような誘電率の時間変動$\varepsilon(t)$が得られる。
			\begin{equation}
				\varepsilon(t)=\varepsilon_{in}+\Delta\varepsilon\left(1-\exp\left(-\frac{t}{\tau}\right)\right) \ \ \  
				\left(\varepsilon(0)=\varepsilon_{in}, \Delta \varepsilon \coloneqq \frac{N\mu^2}{kT}\right)
			\end{equation}
			ここで$\varepsilon_{in}$は瞬間誘電率であり、配向分極より短時間で表れる電子分極、イオン分極の誘電率に対応する。
			周波数おいて、配向分極より高周波に対応するため$\varepsilon_{\infty}$と記述される場合もある。
			パルス電場が印加されたとすると式\ref{パルス応答関数}を用いて以下の誘電率$\varepsilon_{p}$が測定される。
			\begin{equation}
				\varepsilon_p(t)=\frac{\partial \varepsilon(t)}{\partial t}
				=\varepsilon_{in}\delta(t)+\frac{\Delta \varepsilon}{\tau}e^{-t/\tau}
				\label{デバイ関数のパルス応答関数}
			\end{equation}
			式\ref{周波数応答関数とパルス応答関数のフーリエ変換}に代入すると以下の周波数応答関数を得る。
			\begin{equation}
				\varepsilon(\omega)=\varepsilon_{\infty}+\frac{\Delta \varepsilon}{1+j\omega \tau }
				\label{デバイ関数}
			\end{equation}
			上式はデバイ関数と呼ばれ、$\omega \tau=1$にて実部は$\Delta \varepsilon$だけ
			減少し、虚部はピークを示す周波数依存性となる。
			実際の物質の誘電緩和スペクトルは式\ref{デバイ関数のパルス応答関数}と式\ref{デバイ関数}は
			より緩やかな時間、あるいは周波数依存性を示す。
			これは微視的に誘電緩和時間が単一ではなく、分布を持っているからである。
			緩和時間の分布を間接的に取り入れ、実測の誘電率の時間依存性を表現する経験式として以下のKorlrausch-Williams-Wattsの式がある。
			\begin{equation}
				\varepsilon(t)=
				\varepsilon_{in}+\Delta \varepsilon \left( 1 - e^{-\frac{t}{\tau}\gamma}\right)
				\ \ \ \ (0\leq \gamma \leq 1)
			\end{equation}
			周波数依存性を再現する経験式として以下のHavriliak-Negamiの式がある。
			\begin{equation}
				\varepsilon(\omega)=
				\varepsilon_{\infty}
				+
				\frac{\Delta \varepsilon}{(1+(i\omega \tau)^\beta)^\alpha}
				\label{Havriliak-Negami}
			\end{equation}
			ここでパラメータ$\alpha, \beta$は正で、$\alpha=1, \beta <1$及び、$\alpha<1, \beta=1$
			の場合はそれぞれ、Cole-Coleの式、Davidson-Coleの式と呼ばれる。
			\newpage
			\section{結晶性高分子物質における繊維構造}
			繊維状の高分子物質は微結晶がとある結晶軸の周りに分布し、その軸の向きは
			図\ref{位置球}における$OY$と一定の角度$\phi$を保ちながらその周りに分布する二重回転の構造となっている。
			図\ref{位置球}における$OY$は繊維軸と一致する。
			また、微結晶の$b$軸を延長し、球と交差する点を$B$とし、
			微結晶の分布は$OB$の周りに一様となる。従って、$b$となす角$\rho$である結晶面の
			法線$N$の球における軌跡は図\ref{位置球}の位置球の表面上において円を描く。
			また、微結晶の$b$軸は分布はY軸と一定の角$\phi$をなして$Y$軸周りに一様に分布する。
			よって図\ref{位置球}における帯状部分をつくる。
			このように結晶面の法線の分布を示した球を位置球といい、図\ref{位置球}の帯状部分を網面帯、
			網面帯を描く元の円を網面円という。
			繊維構造は$\phi$ の値により3つに大別される。繊維軸と微結晶の軸が一致する$\phi=0$を単純繊維構造、
			有限の角度を持つ場合$\phi=\phi_0$をらせん繊維構造、
			垂直$\phi=90^{\circ}$を環状繊維構造と呼ぶ。
			単純繊維構造としてはラミー\cite{ラミー}、絹糸\cite{絹糸構造}が挙げられる。
			らせん繊維構造としては木綿が挙げられ、らせん角$\phi \approx 30^{\circ}$である\cite{cotton}。
			\begin{figure}[h]
				\centering
				\includegraphics[scale=0.8]{位置球.jpg}
				\caption{微結晶と位置球の関係. 左が微結晶、右が1/8象限の位置球}
				\label{位置球}
			\end{figure}
			\begin{figure}[h]
				\centering
				\includegraphics[scale=0.8]{fiber_structures.jpg}
				\caption{繊維の構造。 
				(A)木綿\cite{cotton}, 
				(B):クモと絹糸(フィブロイン)の構造\cite{クモと絹糸の構造}}
			\end{figure}
	\chapter{実験手法}
		\section{試料作製方法}
		\newpage
		\section{評価方法}
			\subsection{X線回折($\theta - 2\theta$測定とPole figure測定)}
			本研究では$\theta-2\theta$測定にて
			結晶性を評価し、配向性の評価にはPole figure測定を用いた。
			\subsubsection{$\theta - 2\theta$測定の原理}
			試料の結晶構造の解析のためにXRD(X-Ray Diffraction)の$\theta-2\theta$測定を行った。
			測定には図\ref{XRD_nakajima}に示したRINT-2000(Rigaku Corporation)を使用した。
			ある結晶粒における面間隔$d$の格子面(hkl)が入射X線に対し、式\ref{Bragg}の式を満たすとき回折が生じる。
			\begin{equation}
				2d \sin \theta = n \lambda \ \ \ \ \ \ (n\in \mathbb{Z})
				\label{Bragg}
			\end{equation}
			このとき、回折線の方向は入射X線の方向に対して図\ref{XRD_theta_2theta}の関係がある。
			よって測定時には図\ref{集中法光学系}のようにX線源とX線検出器を走査させる。
			図\ref{XRD_theta_2theta}の赤い矢印は回折に寄与している格子面の法線方向を示している。
			格子面の法線方向、もしくは逆格子ベクトル$g_{hkl}$は測定時において入射角$\theta$を走査させても変動しない。
			よって$\theta-2\theta$測定は測定対象の全ての結晶の格子面の法線方向が測定試料の法線方向に存在している状態が望ましい。
			また、この状態を多結晶という。結晶が配向している、あるいは単結晶の場合において$\theta-2\theta$測定を行うと
			ピークが小さくなる、見れないといった現象を生じる。よって本研究では$\theta-2\theta$測定にて
			結晶性を評価し、配向性の評価にはPole figure測定を用いた。
			\begin{figure}[h]
				\centering
				\begin{minipage}{0.45\hsize}
					\centering
					\includegraphics[width=0.9\linewidth]{sora.jpg}
					\caption{RINT-2000(Rigaku Corporation)}
					\label{XRD_nakajima}
				\end{minipage}
				\begin{minipage}{0.45\hsize}
					\centering
					\includegraphics[scale=0.9]{theta_2theta.jpg}
					\caption{$\theta-2\theta$測定における面間隔$d$と$\theta$の関係。赤い矢印は回折に寄与している格子面の法線方向。}
					\label{XRD_theta_2theta}
				\end{minipage}
			\end{figure}
			\newpage
			\begin{figure}[h]
				\centering
				\includegraphics[scale=1.3]{XRD_theta_2theta_光学系.png}
				\caption{集中法光学系}
				\label{集中法光学系}
			\end{figure}

			\subsubsection{Pole figure測定の原理}
			$\theta-2\theta$測定は試料表面に平行な格子面を測定するため、
			結晶面の逆格子ベクトル$\bm{g}_{hkl}$は常に試料の表面に垂直方向を向いている。
			逆格子ベクトル$\bm{g}_{hkl}$の大きさは結晶面の間隔$d$の逆数$1/d$に対応しており、
			$\theta-2\theta$測定は結果的に様々な$d$値の格子面を観測する。
			$\theta-2\theta$測定においても試料の配置を変更するなどによって配向の有無程度は判断可能である。
			しかし配向度などの定量な詳しい評価はPole figure測定などが必要になる。
			
			Pole figure測定(別名:極点測定)は$\theta-2\theta$測定における回折角度$2\theta$を固定し、
			試料を回転、あおりを行い回折強度を測定する手法である。測定する回折面は固定されているため図\ref{極点の測定方法}の通り
			、$\bm{g}_{hkl}$の長さは一定となる。試料を$\alpha$(あおり)と$\beta$(面内回転)という
			2つの優先方位軸を用いて、回折角度一定の半球をスキャンする。
			計測結果は図\ref{極点の測定方法}(B)の通り、極図形を用いて表現する。
			配向度は図\ref{極点の測定方法}(C)の通り、得られた極座標を半径方向に積分し、以下の式\ref{orientation}に基づいて計算される。
			\begin{equation}
				配向度 = \frac{360 - \sum ピークの半値幅 }{360}\times 100
				\label{orientation}
			\end{equation}
			\begin{figure}[h]
				\centering
				\includegraphics[width=\linewidth]{極点の測定方法.jpg}
				\caption{Pole figure測定の概念図. (A)はスキャン方向. (B)は極図形. (C)は極図形の$\beta$依存性.}
				\label{極点の測定方法}
			\end{figure}
			\newpage
			Pole figure測定には図\ref{Pole_figure_outplane}のアウトプレーンの測定と図\ref{Pole_figure_inplane}のインプレーンの測定が存在する。
			アウトプレーンは試料に対するあおりを試料台を傾けて実現する。
			よって、$\alpha=0^{\circ}$近傍においては試料の側面に入射されてしまうため測定が困難になる。
			結果的にアウトプレーンの想定では$\alpha=0^{\circ}\sim90^{\circ}$の全極点測定は実質不可能である。
			インプレーンのPole figure測定は図\ref{Pole_figure_inplane}の通り、
			試料台を傾けず受光部を傾けてあおり$\alpha$を実現する。
			光学系においては図\ref{XRD_配置}の$2\theta$軸に直行する軸である$2\theta_\chi$軸を利用する。
			インプレーンのPole figure測定で$\alpha=0$は低角で測定する薄膜法を同様の状況であるため、
			薄膜法で使用する光学素子を使用する。例えば、集中光学法では集中発散ビーム(BB)を使用するが
			Pole figure測定では薄膜法で使用される平行ビーム(PB)を使用する。
			本研究においてはPole figure測定が可能であるSmart lab(Rigaku Corporation)を使用した。
			
			\begin{figure}[h]
				\centering
				\includegraphics[width=0.9\linewidth]{pole_figure_outplane.jpg}
				\caption{アウトプレーンのPole figure測定}
				\label{Pole_figure_outplane}
			\end{figure}
			\begin{figure}[h]
				\centering
				\includegraphics[width=0.9\linewidth]{pole_figure_inplane.jpg}
				\caption{インプレーンのPole figure測定}
				\label{Pole_figure_inplane}
			\end{figure}
			\begin{figure}[h]
				\centering
				\includegraphics[width=0.9\linewidth]{XRD_配置.jpg}
				\caption{XRDの測定系}
				\label{XRD_配置}
			\end{figure}
			\newpage
			\begin{figure}[h]
				\centering
				\includegraphics{sora.jpg}
				\caption{Pole figure測定に用いたXRD(Rigaku corportaion)}
				\label{XRD_pole_figure}
			\end{figure}
			\newpage
			\subsection{誘電率測定と圧電共鳴法}
			作成したシルクフィブロインフィルムの誘電率はインピーダンスアナライザ
			図\ref{インピーダンスアナライザ}(KEYSIGHT 4294A)を用いて測定した。
			圧電性の評価においては誘電率に現れる圧電性による効果を圧電共鳴法に基づいて評価した。
			\begin{figure}[h]
				\centering
				\begin{minipage}{0.45\hsize}
					\centering
					\includegraphics{sora.jpg}
					\caption{KEYSIGHT 4294A}
					\label{インピーダンスアナライザ}
				\end{minipage}
				\begin{minipage}{0.45\hsize}
					\centering
					\includegraphics{自動平衡ブリッジ回路.jpg}
					\caption{自動平衡ブリッジ回路}
					\label{自動平衡ブリッジ回路}
				\end{minipage}
			\end{figure}
			\subsubsection{誘電率測定の原理}
			インピーダンスアナライザは自動平衡ブリッジ回路で構成されており図\ref{自動平衡ブリッジ回路}の通りである。
			自動平衡ブリッジ回路は試料に印加される電圧を計測する交流電圧計$V_1$と、
			試料を流れる電流を算出するために用いる交流電圧計$V_2$から構成されている。
			また、交流電圧計$V_1$は四端子測定が行われている。
			図\ref{自動平衡ブリッジ回路}における$L_C$はオペアンプの仮想短絡により0 Vである。
			オペアンプの高インピーダンス特性により、試料を流れる電流$i_x$と$i_r$は等価とみなせる。
			$i_r$は以下の式の通り、交流電圧計$V_2$を用いて計算できる。
			\begin{equation}
				i_r=i_x=-\frac{V_2}{R_r}
			\end{equation}
			試料に印加される電圧$V_1$と試料を流れる電流$i_x$を用いて以下の式のように、
			試料のインピーダンス$Z$を算出できる。
			また、式\ref{impedance}における$R$はレジスタンス、$X$はリアクタンスである。
			\begin{equation}
				Z=\frac{V_1}{i_x}=\frac{V_1}{i_r}=-\frac{V_1}{V_2}R_r = R + j X  
				\label{impedance}
			\end{equation}
			式\ref{impedance}よりインピーダンス$Z$は基準抵抗$R_r$と交流電圧比$V_1/V_2$から求まる。
			交流電圧比$V_1/V_2$はロックインアンプを用いて計算されるため、
			インピーダンスの大きさ$|Z|=|V_1 R_r/V_2|$と位相(Phase)$\phi$が出力される。
			その二つをもちいるとレジスタンス$R$とリアクタンス$X$の値も以下の式で計算される。
			\begin{equation}
				R = |Z| \cos \phi
			\end{equation}
			\begin{equation}
				X = |Z| \sin \phi
			\end{equation}
			またアドミッタンスはインピーダンスの逆数であるため式\ref{admittance}のように決まる。
			式\ref{admittance}における$G$はコンダクタンス、$B$はサセプタンスである。
			\begin{equation}
				Y= \frac{1}{Z}=-\frac{V_2}{V_1}\frac{1}{R_r} = G + j B
				\label{admittance}
			\end{equation}
			コンダクタンス$G$、サセプタンス$B$はロックインアンプにより出力される
			インピーダンスの大きさ$|Z|$と位相$\phi$を用いて以下のように計算される。
			\begin{equation}
				G = \frac{1}{|Z|} \cos \left(-\phi\right) = \frac{1}{|Z|} \cos \phi
				\label{コンダクタンス}
			\end{equation}
			\begin{equation}
				B = \frac{1}{|Z|} \sin(-\phi) = \frac{-1}{|Z|}\sin \phi
				\label{サセプタンス}
			\end{equation}

			測定試料が高インピーダンスな誘電体であり、
			測定する周波数区間が40 Hz -110 MHzであるため、
			測定試料の等価回路として抵抗$R_s$とコンデンサ$C_s$の並列モデルを採用する。
			このとき、アドミッタンス$Y_s$は以下のように記述できる。
			\begin{equation}
				Y_s = \frac{1}{R_s}+j\omega C_s
				\label{Y_s}
			\end{equation}
			一方、測定試料を理想的なコンデンサとみなすとき、そのアドミッタンスは試料の寸法
			(厚さ$d$, 電極面積$S$)を用いて以下のように求められる。
			\begin{equation}
				Y_p= j \omega C_p = j \omega \varepsilon^{*} \frac{S}{d} = 
				j\omega (\varepsilon'-j\varepsilon'')\frac{S}{d} =
				\omega\varepsilon''\frac{S}{d} + j\omega \varepsilon' \frac{S}{d}
				\label{Y_p}
			\end{equation}
			式\ref{Y_s}の$Y_s$と式\ref{Y_p}の$Y_p$は等価であるため、
			誘電率の実部$\varepsilon'$と虚部$\varepsilon''$は以下のように記述できる。
			\begin{equation}
				\varepsilon' = C_s \frac{d}{S}
				\label{real_epsilon}
			\end{equation}
			\begin{equation}
				\varepsilon'' = \frac{d}{\omega R_s S}
				\label{imaginary_epsilon}
			\end{equation}
			これより、インピーダンスアナライザを用いてインピーダンスの大きさ$|Z|$
			と位相$\phi$を計測し
			式\ref{コンダクタンス}、式\ref{サセプタンス}、式\ref{Y_s}から
			抵抗$R_s$とコンデンサ$C_s$の値を求める。
			そして式\ref{real_epsilon}と式\ref{imaginary_epsilon}から
			誘電率の実部虚部を計算する。
			
			高周波においては誘電体のインピーダンスが小さくなる。
			誘電体表面に作製した電極のインピーダンスが、誘電体のインピーダンスに対して相対的に影響が大きくなる。
			等価モデルはコンデンサと抵抗の並列モデルではなく、抵抗とコンデンサの直列モデルが妥当となる場合がある。
			高周波においてもインピーダンスアナライザを用いた計測を行う場合、
			電極の抵抗を低くするなど工夫が必要である。
			
			本研究において圧電性の評価は誘電率を測定し、誘電率に現れる圧電共鳴成分を用いて評価
			する圧電共鳴法を用いた。
			\newpage
			\subsubsection{圧電共鳴法}
			圧電体は、逆圧電効果によって電圧を印加すると機械的な変形を生じる。
			交流電圧を印加すると特定の周波数において機械的な共鳴状態が発生する。
			これを圧電共鳴と呼ぶ。
			PVDFやVDF-TrFEなどの長さ方向、幅方向、厚さ方向に圧電性を持つ材料は
			図\ref{圧電共鳴のモード}のように3モードにおいて共振する。
			長さ方向の共振においては、長さ、幅、厚さの全てが自由端条件となる。
			幅方向の共振は、長さ方向が固定端条件となりそれ以外は自由端条件である。
			厚さ方向の共振は、厚さ方向のみが自由端条件となる。
			そしてそれぞれのモードにおいて、誘電率においても
			図\ref{VDF-TrFEの圧電共鳴スペクトル}のように共鳴のスペクトルを生じる。
			\begin{figure}[h]
				\centering
				\includegraphics[width=0.5\linewidth]{圧電共鳴.jpg}
				\caption{長さ方向、幅方向、厚さ方向に圧電生を持つ材料の圧電共鳴。(a):長さ方向の共鳴、(b):幅方向の共鳴、(c):厚さ方向の共鳴。
				$l$は試料の長さ方向の寸法、$w$は幅方向の寸法、$t$は厚さ方向の寸法。}
				\label{圧電共鳴のモード}
			\end{figure}
			\begin{figure}[h]
				\centering
				\includegraphics[width=0.5\linewidth]{VDF_TrFe_圧電共鳴.jpg}
				\caption{VDF-TrFEの圧電共鳴スペクトル\cite{小林理研ニュース}}
				\label{VDF-TrFEの圧電共鳴スペクトル}
			\end{figure}
			\\ \\
			長さ方向の共鳴が発生する周波数を$f_l$、幅方向の共鳴の周波数を$f_w$、
			厚さ方向の共鳴の周波数を$f_t$はそれぞれの寸法を$l, w, t$として
			以下の3式で表される。
			\begin{equation}
				f_l = \frac{1}{2l}\sqrt{\frac{1}{\rho s_l}}
			\end{equation} 
			\begin{equation}
				f_w = \frac{1}{2w}\sqrt{\frac{1}{\rho s_w}}
			\end{equation}
			\begin{equation}
				f_t = \frac{1}{2t}\sqrt{\frac{c_t}{\rho}}
			\end{equation} 
			また、電気機械結合係数は0°カットであれば、それぞれ方向において以下のように決まる。
			\begin{equation}
				k_l^2 = \frac{d_{31}^2}{\varepsilon_{33}^T s_{11}}
			\end{equation}
			\begin{equation}
				k_w^2 = \frac{\left(d_{32} - d_{31}s_{12}/s_{22} \right)^2}
				{\varepsilon_{33}^T s_{22}}
			\end{equation}
			\begin{equation}
				k_t^2 = \frac{e_{33}^2}{\varepsilon_{33}^S c_{33}}
			\end{equation}
			誘電率全体を上の3式を用いて以下の様に記述される。
			$\varepsilon_{33}^S$は3軸どの方向においてもひずみがない、
			あるいは固定端であるときの誘電率であり、式\ref{Havriliak-Negami}のHavriliak-Negamiの式と同値である。
			\begin{equation}
				\varepsilon_{33} =
				\varepsilon_{33}^S
				\frac{\left(1+\frac{k_l^2}{1-k_l^2}\frac{\tan{\pi f/2f_l}}{\pi f/2f_l}\right)
				\left( 
					1+\frac{k_w^2}{1-k_w^2}\frac{\tan{\pi f/2f_w}}{\pi f/2f_w}
				\right)
				}
				{1-k_t^2\frac{\tan{\pi f/2f_t}}{\pi f/2f_t}}
				\label{normal_piezo_resonance}
			\end{equation}
			\\
			ずり圧電の共鳴モードは図\ref{圧電共鳴のモード}と異なる。
			ずり圧電の場合は、図\ref{ずり圧電の共鳴モード}(a)の様に
			配向軸に対して$\theta=45^\circ$にて正方形にカットした場合にて
			図\ref{ずり圧電の共鳴モード}(b)の共鳴モードとなる。
			正方形であるため、共鳴モードは一つしかない。
			どの試料の辺においても境界条件は自由端となる。
			また、試料の厚さは長さや幅に対して十分小さいとみなして固定端条件となる。
			この条件は、PVDFやVDF-TrFEなどにおける、厚さ方向の共鳴と同じである。
			また、ずり圧電は慣例的に膜厚方向を1軸、配向軸に垂直な方向をy軸、配向軸に平行な方向をz軸とる。
			ずり圧電の性質をもつ材料としてPLLA(ポリ乳酸)がある。
			PLLAにおける圧電スペクトルは図\ref{PLLA_piezo_resonance}の通りである。
			\begin{figure}[h]
				\centering
				\includegraphics[width=0.8\linewidth]{ずり圧電_共鳴モード.jpg}
				\caption{ずり圧電の共鳴モード。(a):配向軸に対して$\theta=45^\circ$の角度で正方形に切り出す。
				膜厚方向をx軸、試料平面内にy, z軸を取る。
				(b):ずり圧電のイメージ図。}
				\label{ずり圧電の共鳴モード}
			\end{figure}
			\begin{figure}[h]
				\centering
				\includegraphics{PLLA_圧電共鳴スペクトル.jpg}
				\caption{ずり圧電の性質をもつPLLAの圧電共鳴スペクトル\cite{plla_piezo_resonance}}
				\label{PLLA_piezo_resonance}
			\end{figure}
			共鳴周波数$f_R$、誘電率の周波数特性$\varepsilon_{11}^T$、電気機械結合係数$k_{14}$は式\ref{ずりの共鳴周波数}、
			式\ref{ずりの圧電共鳴_理論式}、式\ref{ずりの電気機械結合係数}で記述される。
			式\ref{ずりの圧電共鳴_理論式}における$\varepsilon_{11}^S$は
			式\ref{normal_piezo_resonance}同様に式\ref{Havriliak-Negami}のHavriliak-Negamiの式と同値である。
			\begin{equation}
				f_R = \frac{1}{2l}\sqrt{\frac{2c_{44}}{\rho}}
				\label{ずりの共鳴周波数}
			\end{equation}
			\begin{equation}
				\varepsilon_{11}^T=\varepsilon_{11}^S\left(1+k_{14}^2\frac{\tan{\pi f/2f_R}}{\pi f/2f_R}\right)
				\label{ずりの圧電共鳴_理論式}
			\end{equation}
			\begin{equation}
				k_{14}^2=\frac{e_{14}^2}{\varepsilon_{11}^Sc_{44}}
				\label{ずりの電気機械結合係数}
			\end{equation}
			式\ref{ずりの圧電共鳴_理論式}は式\ref{normal_piezo_resonance}において、
			$k_l=0, k_w=0$とし、$t\rightarrow l, f_t \rightarrow f_R$にして以下のように近似すると得られる。
			\begin{equation}
				\varepsilon_{33} =
				\varepsilon_{33}^S
				\frac{1}{1-k_l^2\frac{\tan{\pi f/2f_R}}{\pi f/2f_R}}
				= \varepsilon_{33}^S\left(1-k_l^2\frac{\tan{\pi f/2f_R}}{\pi f/2f_R}\right)^{-1}
				\approx \varepsilon_{33}^S\left(1+k_l^2\frac{\tan{\pi f/2f_R}}{\pi f/2f_R}\right)
			\end{equation}
			図\ref{PLA_45_cut}のように試料を正方形ではなく、
			長方形にカットして圧電共鳴スペクトルを観測した事例もある
			\cite{PLA_長方形_圧電共鳴}。
			ただし、一般的にずり圧電$k_{14}$は微小で共鳴スペクトルが明確でない場合もあるが、
			長さ方向と幅方向のどちらにおいても計測対象が$k_{14}$の1つであるため
			本研究では正方形カットを採用した。
			\begin{figure}[h]
				\centering
				\begin{minipage}{0.45\hsize}
					\centering
					\includegraphics{PLA_ずり圧電_長方形カット.jpg}
					\caption{PLAの長方形$45^\circ$カット\cite{PLA_長方形_圧電共鳴}}
					\label{PLA_45_cut}
				\end{minipage}
				\begin{minipage}{0.45\hsize}
					\centering
					\includegraphics{PLA_圧電共鳴スペクトル_長方形.jpg}
					\caption{PLAの長方形$45^{\circ}$カットにおける圧電共鳴スペクトル観測\cite{PLA_長方形_圧電共鳴}}
				\end{minipage}
			\end{figure}
			\newpage
			\subsection{DSC}
			\subsection{FTIR}
	\chapter{結果と考察}
	\chapter{総括}
	\chapter{付録}
	\subsection{原子間力顕微鏡(AFM), 圧電応答顕微鏡(PFM)}
	原子間力顕微鏡(AFM)、圧電応答顕微鏡(PFM)はどちらも走査型プローブ顕微鏡(SPM)に属する。
	走査型プローブ顕微鏡(SPM)とは、微小な深針(カンチレバー)で試料をなぞり、
	その形状や物性を観察、計測する顕微鏡の総称である。図\ref{SPM}, \ref{cantilever_photo_detector}
	にSPMの基本構成を示す。SPMに属する顕微鏡は図\ref{SPM}のように
	カンチレバーを試料表面に接触または接近させて、走査中に生じる試料のある物理量の変化を検出する。
	本研究にて用いるSPMは原子間力顕微鏡(AFM)と圧電応答顕微鏡(PFM)のみである。
	この二つにおいては、図\ref{cantilever_photo_detector}のようにそれぞれの物理量変化
	によってカンチレバーがたわみ、そのカンチレバーに照射したレーザーの変位を
	フォトディテクターで計測し、測定対象の物理量変化を測定する。
	つまり、レーザーの変位から物質表面の物性変化をみる。
	原子間力顕微鏡(AFM)はカンチレバーと試料間に生じる原子間力を検出し
	試料表面画像を取得する。圧電応答顕微鏡(PFM)は試料の逆圧電効果に
	よる表面の歪みをカンチレバーの変位として取得し画像化する。
	\begin{figure}[h]
		\centering
		\begin{minipage}{0.45\hsize}
			\centering
			\includegraphics[width=\linewidth]{SPM.png}
			\caption{走査型顕微鏡(SPM)の光学系}
			\label{SPM}
		\end{minipage}
		\begin{minipage}{0.45\hsize}
			\centering
			\includegraphics[width=\linewidth]{SPM2.png}
			\caption{CantileverとPhotodetectorの関係}
			\label{cantilever_photo_detector}
		\end{minipage}
	\end{figure}
		\subsubsection{AFM(Atomic Force Microscopy)}
		\begin{figure}[h]
			\centering
			\includegraphics[width=0.8\linewidth]{AFM.png}
			\caption{AFMの信号模式図}
			\label{stracture_of_AFM}
		\end{figure}
		図\ref{stracture_of_AFM}はAFMの信号模式図である。AFMの測定手法にはタッピングモード(ACモード)
		とコンタクトモード(DCモード)がある。タッピングモードは、カンチレバーを内部に存在する圧電素子を用いて
		共振周波数で振動させ、試料表面を断続的に接触させながら走査する。測定対象の表面形状からカンチレバーの振動振幅が変動し、
		画像化する。正確にはカンチレバーを試料表面に近づけると原子間力を検出し、
		この瞬間カンチレバーの振動振幅は小さくなる。この振動振幅の変位をレーザーの変位として取得し、
		この変位分だけ元に戻すようにフィードバックをかけ、Z軸方向をZピエゾで調整する。このZピエゾの変位を画像化し、表面像を得る。
		コンタクトモードはタッピングモードと異なり、
		カンチレバーを振動させずに静的な状態で試料に常に接触させながら試料表面を走査し、
		表面の凹凸に対応したカンチレバーのたわみをレーザーの変位としてフォトディテクターから検出する。
		このレーザーの変位を一定にするようにZピエゾを用いて、フィードバック制御を行う。そのZピエゾの変位を表面形状として画像化する。
		本研究において、AFMを用いた計測はカンチレバーの消耗の観点からタッピングモードで行った。
		\subsubsection{PFM(Piezoresponse Force Microscopy)}
		走査型プローブ顕微鏡(Scanning Prob Microscope, SPM)の一種として圧電応答顕微鏡(Piezoresponse Force Microscopy, PFM)がある。
		圧電応答顕微鏡(PFM)は試料の逆圧電効果による表面の歪みをカンチレバーの変位として取得し画像化する顕微鏡である。
		図\ref{PFMの概念図}のように試料片面基板からカンチレバー間で交流電場を印加する。
		このとき試料の分極方向に依存して、電場の変化に対応した歪みが試料に生じる。
		その歪の大きさはカンチレバーの変位の大きさとして現れ、
		Amplitude像として取得できる。
		\begin{figure}[h]
			\centering
			\includegraphics{PFM.png}
			\caption{PFMの概念図}
			\label{PFMの概念図}
		\end{figure}
		\\
		また、図\ref{PFM_phaseの概念図}のように交流電場に対応した試料の歪み方向は試料中の分極方向に依存しているため、
		この分極方向を交流電場と歪みの位相差から解析し、Phase像として取得できる。
		\begin{figure}[h]
			\centering
			\includegraphics[width=0.8\linewidth]{PFM_phase.png}
			\caption{PFMのPhase像概念図。(a):電場E=0,(b):交流電場と試料の伸縮方向が同位相であるときの自発分極方向,(c):交流電場と試料の伸縮方向が逆位相であるときの自発分極方向}
			\label{PFM_phaseの概念図}
		\end{figure}
		PFMはカンチレバーを試料上に接触させ走査するコンタクトモードで行われる。
		カンチレバーと試料からの相互作用を加味した共振周波数、
		コンタクト周波数でカンチレバーを振動させることで、
		その振動振幅の変化から、より高精度に圧電応答を観察できるようになっている。
		しかし、そのコンタクト周波数は走査中一定に保たれているわけではない。
		なぜならば、走査中の試料表面の形状変化によるカンチレバーの接触面積の違いがコンタクト周波数をシフトさせる要因になる。
		コンタクト周波数のシフトにより、圧電応答によるカンチレバーのたわみ変化の大きさと走査中における試料表面の形状変化によるカンチレバーのたわみ変化がクロストークしてしまい、
		正確な圧電応答や表面像を取得できなくなる。よって測定材料は測定前に表面を平らにする必要がある。
		
		図\ref{PFM_phaseの概念図}は垂直方向のみでの議論である。しかし、図\ref{PFMの概念図}におけるフォトディテクターから分かる通り、
		図\ref{面内と垂直の概念図}の様に試料の面内方向の圧電性の評価も可能である。本研究においてはずり圧電の評価にて使用した。
		\begin{figure}[h]
			\centering
			\includegraphics[scale=0.7]{lateral_vertical.png}
			\caption{カンチレバーにおける面内方向(Lateral)と垂直方向(Vertical)の関係}
			\label{面内と垂直の概念図}
		\end{figure}
		\newpage
		生体材料の一つであるコラーゲンの圧電行列は以下の通りである。
		\begin{equation}
			\left(
		\begin{array}{cccccc}
			0&0&0&d_{14}&d_{15}&0 \\
			0&0&0&d_{15}&-d_{14}&0 \\
			d_{31}&d_{31}&d_{33}&0&0&0
		\end{array}\right)
	\end{equation}
		コラーゲン繊維の側面において面内PFM、コラーゲン繊維の断面において垂直PFMを行った報告がある\cite{コラーゲン繊維PFM}。
		このように生体材料の圧電測定においてPFMは頻繁に利用されている。
		\begin{figure}[ht]
			\begin{center}
			\includegraphics[scale=0.6]{PFM_コラーゲン_側面.jpg}
			\caption{コラーゲン繊維(側面)での面内PFM\cite{コラーゲン繊維PFM}.
			(A)AFM topology(B)面内PFM(C)面内PFMのPhase
			(D)AFM topology(E)面内PFM(F)面内PFMのPhase.
			(A)-(C)のスケールバーは2 {\textmu}m, 
			(D)-(F)のスケールバーは200 nm.}
			\label{PFM_コラーゲン_側面}
			\end{center}
		\end{figure}
		\begin{figure}[h]
			\centering
			\includegraphics[scale=0.6]{PFM_コラーゲン_断面.jpg}
			\caption{コラーゲン繊維(断面)での垂直PFM\cite{コラーゲン繊維PFM}.
			(A)AFM topology(B)垂直PFM(C)垂直PFMのPhase
			(D)AFM topology(E)垂直PFM(F)垂直PFMのPhase.
			(A)-(C)のスケールバーは2 {\textmu}m, 
			(D)-(F)のスケールバーは200 nm.}
			\label{PFM_コラーゲン_断面}
		\end{figure}
	\bibliography{master} %hoge.bibから拡張子を外した名前
	\bibliographystyle{junsrt} %参考文献出力スタイル
	\chapter*{謝辞}
	
\end{document}